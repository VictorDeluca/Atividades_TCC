\title{Monografia TCC}

\documentclass[
	% -- opções da classe memoir --
	12pt,				% tamanho da fonte
	%openright,			% capítulos começam em pág ímpar (insere página vazia caso preciso)
	%twoside,			% para impressão em verso e anverso. Oposto a oneside
  oneside,
	a4paper,			% tamanho do papel.
	% -- opções da classe abntex2 --
	chapter=TITLE,		% títulos de capítulos convertidos em letras maiúsculas
	%section=TITLE,		% títulos de seções convertidos em letras maiúsculas
	%subsection=TITLE,	% títulos de subseções convertidos em letras maiúsculas
	%subsubsection=TITLE,% títulos de subsubseções convertidos em letras maiúsculas
	% -- opções do pacote babel --
	english,			% idioma adicional para hifenização
	% french,				% idioma adicional para hifenização
	% spanish,			% idioma adicional para hifenização
	brazil				% o último idioma é o principal do documento
	]{abntex2}

  %\renewcommand{\ABNTEXpartfontsize}{\normalsize}
	%\renewcommand{\ABNTEXchapterfontsize}{ \large}
	%\renewcommand{\ABNTEXsectionfontsize}{\normalsize}
	%\renewcommand{\ABNTEXsubsectionfontsize}{\normalsize}


% ---
% Pacotes básicos
% ---
\usepackage{lmodern}			% Usa a fonte Latin Modern
\usepackage[T1]{fontenc}		% Selecao de codigos de fonte.
\usepackage[utf8]{inputenc}		% Codificacao do documento (conversão automática dos acentos)
\usepackage{lastpage}			% Usado pela Ficha catalográfica
\usepackage{indentfirst}		% Indenta o primeiro parágrafo de cada seção.
\usepackage{color}				% Controle das cores
\usepackage{graphicx}			% Inclusão de gráficos
\usepackage{svg}
\graphicspath{{resources/}}  	% Caminho para as imagens
\usepackage{microtype} 			% para melhorias de justificação
\usepackage{todonotes}
\usepackage{tabularx}
\renewcommand{\tabularxcolumn}[1]{m{#1}}
\usepackage{xcolor}
\usepackage{url}

\usepackage{listings}
\definecolor{algColor}{RGB}{255,206,206} % rgb(255, 206, 206)

%

% ---
% Pacotes adicionais, usados apenas no âmbito do Modelo Canônico do abnteX2
% ---
\usepackage{lipsum}				% para geração de dummy text
% ---

% ---
% Pacotes de citações
% ---
\usepackage[brazilian,hyperpageref]{backref}	 % Paginas com as citações na bibl
\usepackage[alf,       % Citações Autor-Data
    abnt-etal-cite=2,  % ``et. al'' a partir de 3 autores nas citações
    abnt-etal-list=3,  % tirar ``et. al'' a partir de 4 autores nas referências
    abnt-emphasize=bf  % Enfatizar o título da publicação com negrito
]{abntex2cite}	% Citações padrão ABNT


% ---
% CONFIGURAÇÕES DE PACOTES
% ---

\usepackage{abntex_ufrr_dcc}

% ---
% Configurações do pacote backref
% Usado sem a opção hyperpageref de backref
\renewcommand{\backrefpagesname}{Citado na(s) página(s):~}
% Texto padrão antes do número das páginas
\renewcommand{\backref}{}
% Define os textos da citação
\renewcommand*{\backrefalt}[4]{
	\ifcase #1 %
		Nenhuma citação no texto.%
	\or
		Citado na página #2.%
	\else
		Citado #1 vezes nas páginas #2.%
	\fi}%
% ---

% ---
% Informações de dados para CAPA e FOLHA DE ROSTO
% ---
\titulo{Uma Abordagem para Geração de Invariantes de Programas Baseado em Templates para Verificação Eficiente de Programas em C}

\autor{VICTOR DELUCA ALMIRANTE GOMES}
\local{Boa Vista - RR}
\data{2018}
\orientador{DSc. Herbert Oliveira Rocha}

\tipotrabalho{Monografia}

\preambulo{Trabalho de conclusão de curso na área de Verificação de Software com o objetivo de melhorar a eficiência dos métodos de verificação de software através da geração de invariantes de programas}
% ---

%-- Informações de dado para a FOLHA DE APROVAÇÃO

\renewcommand{\orientadorBanca}{Prof. DSc. Herbert Oliveira Rocha}
\renewcommand{\primeiroMembroBanca}{Prof. DSc. Luciano Ferreira Silva}
\renewcommand{\segundoMembroBanca}{Prof. MSc. Miguel Raymundo Flores Santibañez}

% ---
% Configurações de aparência do PDF final

% alterando o aspecto da cor azul
\definecolor{blue}{RGB}{41,5,195}

% informações do PDF
\makeatletter
\hypersetup{
     	%pagebackref=true,
		pdftitle={\@title},
		pdfauthor={\@author},
    	pdfsubject={\imprimirpreambulo},
	    pdfcreator={LaTeX with abnTeX2},
		pdfkeywords={abnt}{latex}{abntex}{abntex2}{trabalho acadêmico},
		colorlinks=true,       		% false: boxed links; true: colored links
    	linkcolor=blue,          	% color of internal links
    	citecolor=blue,        		% color of links to bibliography
    	filecolor=magenta,      	% color of file links
		urlcolor=blue,
		bookmarksdepth=4
}
\makeatother
% ---

% ---
% Espaçamentos entre linhas e parágrafos
% ---

% O tamanho do parágrafo é dado por:
\setlength{\parindent}{1.3cm}

% Controle do espaçamento entre um parágrafo e outro:
\setlength{\parskip}{0.2cm}  % tente também \onelineskip

% ---
% compila o indice
% ---
\makeindex
% ---

% ----
% Início do documento
% ----
\begin{document}

% Retira espaço extra obsoleto entre as frases.
\frenchspacing

% ----------------------------------------------------------
% ELEMENTOS PRÉ-TEXTUAIS
% ----------------------------------------------------------
% \pretextual

% ---
% Capa
% ---
\imprimircapa
% ---

% ---
% Folha de rosto
% (o * indica que haverá a ficha bibliográfica)
% ---
\imprimirfolhaderosto
% ---

% ---
% Inserir folha de aprovação
% --- \imprimirfolhadeaprovacao
% ---
% Dedicatória
% ---
\begin{comment}
\begin{dedicatoria}
   \vspace*{\fill}
   \centering
   \noindent
   \textit{ Dedicatória }
   \vspace*{\fill}
\end{dedicatoria}
% ---

% ---
% Agradecimentos
% ---
\begin{agradecimentos}[agradecimentos]

Agradeço a todas as pessoas que contribuíram, direta ou indiretamente, com a realização deste trabalho.

\end{agradecimentos}
\end{comment}
% ---

% ---
% Epígrafe
% ---
\begin{epigrafe}
    \vspace*{\fill}
	\begin{flushright}
		\textit{``Enjoy the pain, it's yours for a while, girl;
            Don't run away, your fear, you're getting older;
            We are relying on you and your emotion;
            And as you go, promise of being bright'' --Veela.}
	\end{flushright}
\end{epigrafe}
% ---

% ---
% RESUMOS
% ---

% resumo em português
\setlength{\absparsep}{18pt} % ajusta o espaçamento dos parágrafos do resumo
\begin{resumo}
A verificação de software é parte vital do processo de desenvolvimento de software, especialmente em sistemas críticos onde erros podem ter consequências catastróficas. Contudo, verificar a corretude de um programa é um processo complexo: Há um trade-off enorme entre velocidade e precisão, e diversos trabalhos no meio acadêmico buscam formas de equilibrar ambos os aspectos. Neste trabalho, é apresentado um método para a melhoria da velocidade dos métodos de verificação formal sem perda de precisão, através do uso de invariantes geradas por templates, que representam de forma precisa a natureza do programa, facilitando o processo de verificação.

\textbf{Palavras-chaves}: Invariantes de Programas; Pré e Pós Condições; Verificação de Programas em C.
\end{resumo}

% ---
% inserir lista de ilustrações
% ---
\pdfbookmark[0]{\listfigurename}{lof}
\renewcommand{\listfigurename}{Lista de Figuras}
\listoffigures*
\cleardoublepage
% ---

% ---
% inserir lista de tabelas
% ---
\pdfbookmark[0]{\listtablename}{lot}
\renewcommand{\listtablename}{Lista de Tabelas}
\listoftables*
\cleardoublepage
% ---

% ---
% inserir lista de abreviaturas e siglas
% ---
% \begin{siglas}
% 	\item[SE] Sistema Embarcado
%     \item[HMM] Hidden Markov Model
%     \item[CAD] Computer Assisted Design
% \end{siglas}
% ---

% ---
% inserir lista de símbolos
% ---
%\begin{simbolos}
%  \item[$ \Gamma $] Letra grega Gama
%  \item[$ \Lambda $] Lambda
%  \item[$ \zeta $] Letra grega minúscula zeta
%  \item[$ \in $] Pertence
%\end{simbolos}
% ---

% ---
% inserir o sumario
% ---
\pdfbookmark[0]{\contentsname}{toc}
\renewcommand{\contentsname}{Sumário}
\tableofcontents*
\cleardoublepage
% ---



% ----------------------------------------------------------
% ELEMENTOS TEXTUAIS
% ----------------------------------------------------------
\textual
% ----------------------------------------------------------
% Introdução
% ----------------------------------------------------------
\chapter{INTRODUÇÃO}
\input{sections/introducao.tex}

% ----------------------------------------------------------
% Revisão de Literatura
% ----------------------------------------------------------
\chapter{CONCEITOS E DEFINIÇÕES}
\input{sections/conceitos_e_definicoes.tex}

\chapter{TRABALHOS CORRELATOS}
\input{sections/trabalhos_correlatos.tex}

% ----------------------------------------------------------
% Detalhes de Desenvolvimento do Projeto
% ----------------------------------------------------------
\chapter{MÉTODO PROPOSTO}
\input{sections/metodo_proposto.tex}

% ----------------------------------------------------------
% Resultados -- Pode vir junto com discussão
% ----------------------------------------------------------
% \chapter{RESULTADOS PARCIAIS}
% \input{sections/resultados_experimentais.tex}
% ----------------------------------------------------------
% Conclusão
% ----------------------------------------------------------
%\chapter{CONSIDERAÇÕES PARCIAIS E PRÓXIMO PASSOS}
%\input{sections/consideracoes_parciais_e_trabalhos_futuros.tex}

% Cronograma
\chapter{Cronograma}
\input{sections/cronograma.tex}

\chapter{Considerações parciais e trabalhos futuros}
\input{sections/consideracoes_parciais_e_trabalhos_futuros.tex}

%\todo[inline]{Falta as considerações parciais}

% ----------------------------------------------------------
% Apêndice 1: Revisão Sistemática da Literatura
% ----------------------------------------------------------
%\chapter{APÊNDICE 1: REVISÃO SISTEMÁTICA DA LITERATURA}
%\input{sections/apendice_1.tex}

% ----------------------------------------------------------
% Referências bibliográficas
% ----------------------------------------------------------

\addto\captionsportuguese{\renewcommand{\bibname}{Reference}}
\renewcommand{\bibname}{Referências}
\bibliography{main}

%---------------------------------------------------------------------
% INDICE REMISSIVO
%---------------------------------------------------------------------
%\phantompart
\printindex
%---------------------------------------------------------------------

\end{document}
